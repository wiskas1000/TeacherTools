\begin{frame}
\frametitle{Formule invullen}
\begin{exampleblock}{Lineair verband}
Gegeven: $f(x) = 3x - 4$.

Bij elke $x$ hoort een $y$. Welke $y$ hoort er bij $\alert{x = 7}$?
\[
 y = 3 \cdot \alert{7} - 4 = 21 - 4 = 17
\]

Dus $f(7) = 17 $. 
\end{exampleblock}

\begin{block}{Vraag 1}<2->
Welke $y$ hoort er bij $x = 2$?
\end{block}

\end{frame}

\begin{frame}
\frametitle{Formule invullen}
\begin{exampleblock}{Kwadratische verbanden}
Gegeven: $f(x) = 3x^2 - 2x + 5$.

Bij elke $x$ hoort een $y$. Welke $y$ hoort er bij $\alert{x =4}$?
\begin{align*}
 y &= 3 \cdot \alert{4}^2 - 2 \cdot \alert{4} + 5 \\
 &= 3 \cdot 16 - 2 \cdot 4 + 5 \\
 &= 48 - 8 + 5 \\
 &= 45
 \end{align*}

Dus $f(4) = 45 $. 
\end{exampleblock}
\end{frame}

\begin{frame}
\frametitle{Formule invullen}
\begin{block}{Vraag 2}
Gegeven de formule $f(x) = 2x^2 - x + 5$. Welke $y$ hoort er bij:
\begin{enumerate}
 \item $x = 1$
 \item $x = 0$
 \item $x = -2$
 \item $x = 3$
 \item $x = 0,25$
\end{enumerate}
\end{block}
\end{frame}

\begin{frame}
\frametitle{Formule invullen}
\begin{block}{Vraag 2}
Gegeven de formule $f(x) = 2x^2 - x + 5$. Welke $y$ hoort er bij:
\begin{enumerate}
 \item $x = 1$ \\
 $f(1) = 6$
 \item $x = 0$ \\
 $f(0) = 5$
 \item $x = -2$ \\
 $f(-2) = 15$ 
 \item $x = 3$ \\
 $f(3) = 20$
 \item $x = 0,25$  \\
 $f(0,25) = 4,875$?
\end{enumerate}
\end{block}

Punten in grafiek: $(1,6)$, $(0,5)$, $(-2,15)$, $(3,20)$ en $(0,25;4,875)$.
\end{frame}


\begin{frame}
\frametitle{Kwadratische verbanden}
\begin{alertblock}{Algemene formule}
De algemene formule van een kwadratisch verband is 
\[ y = ax^2 + bx + c\]
\end{alertblock}

\begin{block}{Vraag 3}
Wat zijn $a$, $b$ en $c$ bij:
\begin{enumerate}
 \item $f(x) = 4x^2 +7x -12$
 \item $y = -3x^2 +9x + 2$
 \item $ y = x^2 + 1$
\end{enumerate}
\end{block}
\end{frame}

\begin{frame}
\frametitle{Kwadratische vergelijking: grafiek tekenen}
\begin{alertblock}{Grafiek tekenen met 5 punten}
\begin{enumerate}
\item Bepaal $a$, $b$ en $c$
\item Bepaal de top
% met $x_{\text{top}} = \frac{-b}{2a}$ 
\item Bepaal de nulpunten (snijpunten $x$-as)
\item Bepaal het snijpunt met de $y$-as
\item Spiegel het snijpunt met de $y$-as
\end{enumerate}
\end{alertblock}
\end{frame}

\begin{frame}
\frametitle{Kwadratische vergelijking: bepalen van de top}
De algemene formule van een kwadratisch verband is \[y = ax^2 + bx + c\]

\begin{alertblock}{Bepalen top}
De $x$-co\"ordinaat van de top is
\[
 x_{\text{top}} = \frac{-b}{2a}
\]

Als je $x_{\text{top}}$ hebt, dan kun je de bijbehorende $y$-co\"ordinaat ook berekenen. Dit doe je door $x_{\text{top}}$ in te vullen in de formule.
\end{alertblock}
\end{frame}

% \frame{\titlepage}
\begin{frame}
\begin{block}{Opgave 2b}
$g(x) = -2x^2 + 3x + 2$
Dus:
\begin{align*}
 a &= -2\\
 b &= 3 \\
 c &= 2
\end{align*}

Dit geeft:
\begin{align*}
 x_{\text{top}} &= \frac{-b}{2a} = \frac{-3}{2 \cdot -2} = \frac{3}{4}
\end{align*}

Dan krijgen we:
\begin{align*}
 y_{\text{top}} &= -2 \cdot \left(\alert{ \frac{3}{4}} \right)^2 + 3 \cdot \left(\alert{\frac{3}{4}}\right) + 2 = 3,125
\end{align*}
Dus top: $(0,75;3,125)$.
\end{block}
\end{frame}

\begin{frame}
\begin{block}{Opgave 2c}
$h(x) = 2x^2 -4x + 5$
Dus:
\begin{align*}
 a &= 2\\
 b &= -4 \\
 c &= 5
\end{align*}

Dit geeft:
\begin{align*}
 x_{\text{top}} &= \frac{-b}{2a} = \frac{- - 4 }{2 \cdot 2} = 1
 \end{align*}

Dan krijgen we:
\begin{align*}
 y_{\text{top}} &= 2 \cdot 1^2 - 4\cdot 1 + 5 = 3
 \end{align*}
Dus top: $(1,3)$.
\end{block}
\end{frame}

\begin{frame}
\begin{block}{Opgave 2d}
$i(x) = -3,43x^2 +1,76x - 7,574$
Dus:
\begin{align*}
 a &= -3,43\\
 b &= 1,76 \\
 c &= -7,574
\end{align*}

Dit geeft:
\begin{align*}
 x_{\text{top}} &= \frac{-b}{2a} = \frac{- 1,76 }{2 \cdot -3,43} = 0,256559766
 \end{align*}

Dan krijgen we:
\begin{align*}
 y_{\text{top}} &= -3,43 \cdot 0,256559766^2 + 1,76 \cdot 0,256559766 - 7,574 \\&= -7,348227405
 \end{align*}
Dus top: $(0,256559766;-7,348227405)$.
\end{block}
\end{frame}
