\section{Introductie}
% \begin{block}{Planning}
% \begin{itemize}
%     \item Periode van 9 weken, \emph{geen} toetsweek
%     \item Studiewijzer: wordt uitgeprint
%     \item Elke week \'e\'en les
% %     \begin{itemize}
% %     \item 2A: Woensdag 120 min
% %     \item 2B: Woensdag 120 min
% %     \item 2C: Woensdag 120 min
% %     \end{itemize}
%     \item Elke week gemiddeld 60 minuten huiswerk
% \end{itemize}
% \end{block}
% \end{frame}
% 
% \begin{frame}
% \begin{block}{Toetsen}
% De situatie is veranderd!
% \begin{itemize}
%     \item Twee blokken (H5/6/7 en H8/9/10/11)
%     \item Elk `blok' heeft minitoetsjes.
%     \item Aan het einde van een blok hebben we een inzichtstoets.
%     \item We hebben \emph{geen} herkansingen!
% \end{itemize}
% \end{block}
% \end{frame}
% 
% \section{Benodigdheden}
 \begin{frame}{Boek}

In de perioden 5 en 6 gebruiken we het boek Wiskunde voor het MLO, deel~2. 

% \bigskip

\begin{block}{Boek}
\begin{minipage}{.70\textwidth} %
Wiskunde voor het MLO deel 2\\
machten / wortels / foutenleer


\bigskip
J.~Lips, A.~Riemslag

\bigskip
ISBN: 978-90-77423-89-9
\vfill
\end{minipage} %
\begin{minipage}{.25\textwidth} %
\begin{figure}
\includegraphics[width=\textwidth]{./afbeeldingen/wiskunde-voor-het-mlo-2-voorkant} 
\end{figure}
\end{minipage}
\end{block}
\end{frame}

\begin{frame}{Benodigdheden}

\begin{block}{Benodigdheden perioden 5 en 6 (1)}
\begin{itemize}
    \item Boek: \emph{Wiskunde voor het MLO, deel 2}
    \item Schrift A4 ruitjes (10mm)
    \item Rekenmachine (geen telefoon!)
    \item Schrijf- en tekenmateriaal
    \begin{itemize}
    \item Pen, potlood/vulpotlood
    \item Geodriehoek of lineaal
    \item Minimaal 1 kleurtje: \color{red}{rood}, \color{green}{groen}, \color{blue}{blauw}
    \end{itemize}
\end{itemize}
\end{block}
\end{frame}



\begin{frame}{Lesstof}
\begin{block}{Lesstof periode 5}
\begin{itemize}
 \item Hoofdstuk 5
 \item Hoofdstuk 6
 \item Hoofdstuk 7
\end{itemize}
\end{block}
\end{frame}

\begin{frame}{Toetsen}
\begin{block}{Toetsen periode 5}
\begin{itemize}
 \item Minitoets hoofdstuk 5
 \item Minitoets hoofdstuk 6
 \item Minitoets hoofdstuk 7
 \item Eindtoets hoofdstuk 5, 6 en 7
 \end{itemize}
\end{block}
\end{frame}
