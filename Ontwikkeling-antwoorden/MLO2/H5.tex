\section{Vermenigvuldigen met machten}
\begin{antwoord}{2a}
Stap 0: $f(x) = x^2 + 5x - 14$, dus $a = 1$, $b = 5$ en $c = -14$.

\medskip
Stap 2:
\begin{align*}
\xtop &= \frac{-b}{2a} = \frac{-5}{2 \cdot 1} = -2,5 \\
\ytop &= \alert{(-2,5)}^2 + 5 \cdot \alert{(-2,5)} - 14\\ 
&= 6,25 + - 12,5- 14\\
&= -20,25\\
Top &: (-2,5; -20,25)
\end{align*}

\medskip
Stap 1: Snijpunten $x-as$:
\begin{align*}
x_1 &= \frac{-b + \sqrt{b^2 - 4ac}}{2a} \\
&= \frac{-5 + \sqrt{5^2 - 4\cdot 1 \cdot -14}}{2 \cdot 1} \\
&= \frac{-5 + \sqrt{25 - 4 \cdot -14}}{2} \\
&= \frac{-5 + \sqrt{25 + 56}}{2} \\
&= \frac{-5 + \sqrt{81}}{2} = 2\\
y_1 &= 0 \qquad \text{(snijpunt met $x$-as)} \\
S_1 &= (2,0)
\end{align*}
\begin{align*}
x_2 &= \frac{-b - \sqrt{b^2 - 4ac}}{2a} \\
&= \frac{-5 - \sqrt{5^2 - 4\cdot 1 \cdot -14}}{2 \cdot 1} \\
&= \frac{-5 - \sqrt{81}}{2} = - 7\\
y_2 &= 0 \qquad \text{(snijpunt met $x$-as)} \\
S_2 &= (-7,0)
\end{align*}

\medskip
Stap 3: Snijpunt $y$-as is op $(0, c)$, dus snijpunt $(0, -14)$.\\

\medskip
Stap 4: Spiegelen: top is in het rood (symmetrie-as)
\begin{wiskundetabel}{5}{x}{y}{-7 & \phantom{-33}  & \alert{-2,5} & 0 & 2}{ 0 & & \alert{-20,25} & -14 & 0}
\end{wiskundetabel}

We weten $(0, -14)$. Spiegelen langs de symmetrie-as geeft:
\begin{wiskundetabel}{5}{x}{y}{-7 & \alertb{-5}  & \alert{-2,5} & 0 & 2}{ 0 & \alertb{-14} & \alert{-20,25} & -14 & 0}
\end{wiskundetabel}

Stap 5: Random punten invullen als je nog geen 5 punten hebt. Hier is dit niet nodig. 

We krijgen:\\
\begin{tikzpicture}[scale=0.5]
\tkzInit[xmin=-10, xmax=5,ymax=5,ymin=-25]
\tkzGrid
    \tkzLabelX[orig=true,label options={font=\scriptsize}]
    \tkzLabelY[orig=false,label options={font=\scriptsize}]
    \tkzDrawX
    \tkzDrawY
    \tkzDefPoint(-7,0){A}
    \tkzDefPoint(-5,-14){B}
    \tkzDefPoint(-2.5,-20.25){C}
    \tkzDefPoint(0,-14){D}
    \tkzDefPoint(2,0){E}
    \tkzDrawPoints(A, B, C, D, E)
    \draw[scale=1, domain=-7.5:2.5,smooth,variable=\x,blue, line width = 1pt] plot ({\x},{\x*\x+5*\x-14});
\end{tikzpicture}
\end{antwoord}





\begin{antwoord}{3b}
Stap 0: $h(a) = a^2 + 7a + 12$, dus $a = 1$, $b = 7$ en $c = 12$.

\medskip
Stap 2:
\begin{align*}
\xtop &= \frac{- 7}{2 \cdot 1} = \frac{-7}{2} = -3,5 \\
\ytop &= \alert{(-3,5)}^2 + 7 \cdot \alert{(-3,5)} + 12\\ 
&= 12,25 + - 24,5 + 12\\
&= -0,25\\
Top &: (-3,5; -0,25)
\end{align*}

\medskip
Stap 1: Snijpunten $x-as$:
\begin{align*}
x_1 &= \frac{-b + \sqrt{b^2 - 4ac}}{2a} \\
&= \frac{-7 + \sqrt{7^2 - 4\cdot 1 \cdot 12}}{2 \cdot 1} \\
&= -3\\
y_1 &= 0 \qquad \text{(snijpunt met $x$-as)} \\
S_1 &= (-3,0)
\end{align*}
\begin{align*}
x_2 &= \frac{-b - \sqrt{b^2 - 4ac}}{2a} \\
&= \frac{-7 - \sqrt{7^2 - 4\cdot 1 \cdot 12}}{2 \cdot 1} \\
&= -4\\
y_2 &= 0 \qquad \text{(snijpunt met $x$-as)} \\
S_2 &= (-4,0)
\end{align*}

\medskip
Stap 3: Snijpunt $y$-as is op $(0, c)$, dus snijpunt $(0, 12)$.\\

\medskip
Stap 4: Spiegelen: top is in het rood (symmetrie-as)
\begin{wiskundetabel}{5}{x}{y}{\phantom{eee} & -4 & \alert{-3,5}  &  -3 & 0}{ & 0 & \alert{-0,25} & 0 & 12}
\end{wiskundetabel}

We weten $(0, 12)$. Spiegelen langs de symmetrie-as geeft:
\begin{wiskundetabel}{5}{x}{y}{\phantom{eee} & -4 & \alert{-3,5}  &  -3 & 0}{ & 0 & \alert{-0,25} & 0 & 12}
\end{wiskundetabel}


Stap 5: Random punten invullen als je nog geen 5 punten hebt. Hier is dit niet nodig. 

We krijgen:\\
\begin{tikzpicture}[scale=0.5]
\tkzInit[xmin=-10, xmax=5,ymax=5,ymin=-25]
\tkzGrid
    \tkzLabelX[orig=true,label options={font=\scriptsize}]
    \tkzLabelY[orig=false,label options={font=\scriptsize}]
    \tkzDrawX
    \tkzDrawY
    \tkzDefPoint(-7,0){A}
    \tkzDefPoint(-5,-14){B}
    \tkzDefPoint(-2.5,-20.25){C}
    \tkzDefPoint(0,-14){D}
    \tkzDefPoint(2,0){E}
    \tkzDrawPoints(A, B, C, D, E)
    \draw[scale=1, domain=-7.5:2.5,smooth,variable=\x,blue, line width = 1pt] plot ({\x},{\x*\x+7*\x+12});
\end{tikzpicture}
\end{antwoord}



























\newpage
\begin{antwoord}{3}
\begin{enumerate}
\item $g(x) = x^2 - 2x +3$, dus $a = 1$, $b = -2$ en $c = 3$.

\medskip
$\xtop = \frac{-b}{2a} = \frac{2}{2 * 1} = 1 $
\begin{align*}
\ytop &= \alert{1}^2 - 2 \cdot \alert{1} + 3\\ 
&= 1 - 2 +3\\
&= 2\\
T &= (1, 2)
\end{align*}
% 
% \medskip
% Snijpunten $x-as$:
% \begin{align*}
% x_1 &= \frac{-b + \sqrt{b^2 - 4ac}}{2a} \\
% &= \frac{-5 + \sqrt{5^2 - 4\cdot 1 \cdot -14}}{2 \cdot 1} \\
% &= \frac{-5 + \sqrt{25 - 4 \cdot -14}}{2} \\
% &= \frac{-5 + \sqrt{25 + 56}}{2} \\
% &= \frac{-5 + \sqrt{81}}{2} = 2\\
% y_1 &= 0 \qquad \text{(snijpunt met $x$-as)} \\
% S_1 &= (2,0)
% \end{align*}
% \begin{align*}
% x_2 &= \frac{-b - \sqrt{b^2 - 4ac}}{2a} \\
% &= \frac{-5 - \sqrt{5^2 - 4\cdot 1 \cdot -14}}{2 \cdot 1} \\
% &= \frac{-5 - \sqrt{81}}{2} = - 7\\
% y_2 &= 0 \qquad \text{(snijpunt met $x$-as)} \\
% S_2 &= (-7,0)
% \end{align*}
% 
% \item $5^5$
% \item $27^{15}$
% \item $4^{12}$
\end{enumerate}
\end{antwoord}
