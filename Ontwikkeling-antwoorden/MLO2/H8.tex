\begin{antwoord}{15}
\begin{enumerate}
 \item[a)]
\begin{align*}
  3^x &= 200 \\
  x &= \logx{3}{200} \\
  &= \logxfrac{3}{200} = 4,822736302
\end{align*}
 \item[b)]
\begin{align*}
  4^{2x} &= 345 \\
  2x &= \logx{4}{345} \\
  2x &= \logxfrac{4}{345} = 4,215226276\\
  x &= 2,10761318
  \end{align*}
\end{enumerate}
\end{antwoord}

\begin{antwoord}{16}
\begin{enumerate}
 \item[a)]
\begin{align*}
  x &= \logx{4}{6} \\
  &= \logxfrac{4}{6} = 1,29248125
\end{align*}
 \item[b)]
\begin{align*}
  1,87^x &= 48,9 \\
  x &= \logx{1,87}{48,9} \\
  &= \logxfrac{1,87}{48,9} \\
  &= 6,214313109
\end{align*}

 \item[c)]
 \begin{align*}
 \logx{8}{56} &= x \\
 x &= \logxfrac{8}{56} = 1,935784974  \\
 \end{align*}
 \item[d)]
 \begin{align*}
 2,8^{3x} &= 68 \\
 3x &= \logx{2,8}{68} = \logxfrac{2,8}{68} = 4,098123671\\
 x &= \frac{4,098123671}{3} = 1,366041224
 \end{align*}
 \item[e)]
 \begin{align*}
\logx{5}{98^2} &= x = \logx{5}{9604}  \\
 x &= \logxfrac{5}{9604} = 5,697600937
 \end{align*}
 \item[f)]
 \begin{align*}
x &= \logx{0,3}{0,1}  \\
&= \logxfrac{0,3}{0,1}  \\
&= 1,912489289
\end{align*}
 \item[g)]
 \begin{align*}
 3^x &= 50 \\
x &= \logx{3}{50}  \\
&= \logxfrac{3}{50}  \\
&= 3,560876795
\end{align*}
 \item[h)]
 \begin{align*}
x &= \logx{0,9}{0,1}  \\
&= \logxfrac{0,9}{0,1}  \\
&= 21,85434533
\end{align*}
 \item[i)]
 \begin{align*}
x &= \logx{-4}{56}
\end{align*}
Kan niet
 \item[j)]
 \begin{align*}
x &= \logx{333}{0,1}\\
 &= \logxfrac{333}{0,1}\\
 &= -0,396440875
\end{align*}
 \item[k)]
 \begin{align*}
45 \cdot 3,85^x &= 567,0 \cdot 10^3 \\
3,85^x &= \frac{567,0 \cdot 10^3}{45} = 12600 \\
x &= \logx{3,85}{12600} = \logxfrac{3,85}{12600} = 7,00366453\\
\end{align*}
 \item[l)]
 \begin{align*}
x &= \logx{4}{-8}
\end{align*}
Kan niet
 \end{enumerate}
\end{antwoord}

\begin{antwoord}{17}
\begin{enumerate}
  \item[b)] $N(t) = N(0) \cdot g^t$ met
  \begin{align*}
    N(0) &= 250 \\
    g &= 2 \\
    t &= \text{Aantal perioden van 10 uur sinds 08:00}
  \end{align*}
  Dit geeft \[N(t) = 250 \cdot 2^t\]
  \item[a)] Om 18:00 zijn er 10 uren voorbij sinds 08:00. Dit geeft $t = 1$ (want er is 1 periode van 10 uur voorbij).
  Om 18:00 zijn er \[N(1) = 250 \cdot 2^1 = 500\] bacteri\"en per mg.

  \bigskip
  Om 22:00 zijn er 14 uren voorbij sinds 08:00. Dit geeft $t = 1,4$ (want $\frac{14}{10} = 1,4$).
  Om 22:00 zijn er \[N(1) = 250 \cdot 2^{1,4} = 659,7539554 \approx 660 \] bacteri\"en per mg.
  \item[c)]
  \begin{align*}
    N(t) &= 250 \cdot 2^t \\
    20000 &= 250 \cdot 2^t \\
    80 &= 2^t \\
    t &= \logx{2}{80}\\
    &= \logxfrac{2}{80} \approx 6,321928095
    \end{align*}
    Dit betekent dat de salade na $6,321928095 \cdot 10 = 63,21928095$ uren moet worden weggegooid.

    \medskip
    Omrekenen naar minuten: Een uur heeft 60 minuten, dus $0,21928095 $ uren zijn $0,21928095 \cdot 60 = 13,156857$ minuten.

    \bigskip
    Na 63 uren en \textbf{14} minuten kun je de salade weggooien (\'e\'en minuut verder en dan zit je over de grens).
\end{enumerate}
\end{antwoord}

\begin{antwoord}{18}
\begin{align*}
N(t) &= N(0) \cdot g^t \\ 
N(0) &= \SI{120}{\litre}\\
g &= 3\\
t &= \text{aantal perioden van 30 minuten vanaf 09:30}\\
\end{align*}
\[
 N(t) = 120 \cdot 3^t 
\]
\begin{enumerate}
\item Om 12:00 zijn er 5 perioden van 30 minuten voorbij sinds 09:30. Dus $t=5$.
Dit geeft
\begin{align*}
  N(t) &= 120 \cdot 3^5\\ 
  &= 29160 
\end{align*}
\item 
\[
 N(t) = 120 \cdot 3^t 
\]
\item 
\begin{align*}
120 \cdot 3^t &= 1 \\
 3^t &= 0,0083333333\ldots \\
 t &= \logx{3}{0,0083333333\ldots}\\
 &= -4,357762781
\end{align*}
$- 4$ perioden betekent dat je $2$~uur terug in de tijd gaat (van 09:30 terug naar 07:30). $-0,357762781$ perioden van 30 minuten zijn $- 0,357762781 \cdot 30 = -10,73288344$ minuten. 

We gaan dus 2 uur en 11 minuten terug in de tijd vanaf 09:30.
Dit betekent dat de wolk is ontstaan om 07:19.
\item
De wolk is \SI{5}{\litre} om 09:00. Om 11:00, dus 4 perioden erna, is de wolk \SI{60}{\litre}. Na 4 perioden is een gaswolk $g^4$ keer zo groot. Dit betekent: 
\begin{align*}
 5 \cdot g^4 &= 60 \\
 g^4 &= \frac{60}{5} = 12 \\
 g &= \sqrt[4]{12}\\
 &= 1,861209718
\end{align*}
De groeifactor is dus $1,861209718$.
\end{enumerate}
\end{antwoord}


\begin{antwoord}{19}
\begin{align*}
N(t) &= N(0) \cdot g^t \\ 
N(0) &= \SI{0,05}{\mm}\\
g &= 2\\
t &= \text{aantal keren knippen en stapelen}\\
\end{align*}
Dit geeft 
\[N(t) = 0,05 \cdot 2^t
\]
\begin{enumerate}
  \item 
  \begin{align*}
   N(3) &= 0,05 \cdot 2^3 = 0,40 \\
      N(5) &= 0,05 \cdot 2^5 = 1,6 \\
         N(10) &= 0,05 \cdot 2^{10} = 51,2 \\
  \end{align*}
  \item 
  \begin{align*}
N(t) &= 0,05 \cdot 2^t
  \end{align*}
\item \SI{3}{\m} hoog: \SI{3000}{\mm}.
\begin{align*}
0,05 \cdot 2^t  &= 3000\\
2^t  &= 60000\\
t  &= \logx{2}{60000}\\
&= \logxfrac{2}{60000}\\
t&= 15,87267488
\end{align*}
Na 16 keren stapelen past de stapel niet meer op de kamer.
\item \SI{3,844e8}{\m} hoog: \SI{3,488e11}{\mm}.
\begin{align*}
0,05 \cdot 2^t  &= \num{3,488e11}\\
2^t  &= \num{6,9622e12}\\
t  &= \logx{2}{\num{6,9622e12}}\\
&= \logxfrac{2}{\num{6,9622e12}}\\
t&= 42,6626804
\end{align*}
Na 43 keren stapelen bereik je de maan.


 \end{enumerate}
\end{antwoord}

