\documentclass[12pt]{report}
% \usepackage{TCR-uitwerkingen}
% \usepackage{arev}
\usepackage{./opmaak/uitwerkingen}
% \author{Wikash Sewlal}
% 
% % \documentclass[12pt]{article}
% % \usepackage[a4paper]{geometry}
% \usepackage{arev}
% 
% \usepackage{booktabs}
% \usepackage{amsmath}
% \usepackage{parskip}
% 
% \usepackage{siunitx}
% 
% \newcommand{\xtop}{\ensuremath{x_{top}}}
% \newcommand{\ytop}{\ensuremath{y_{top}}}
% 
% \author{Wikash Sewlal}
% % \newenvironment{antwoord}[1]{\subsection{Antwoord opgave #1} }{}
% \usepackage{enumitem}
% \setlist[enumerate]{label=\alph*),itemsep=3pt,topsep=3pt}
% 
% % \newcommand{\logx}[2]{\ensuremath{^{#1}\log(#2)}}
% \newcommand{\logx}[2]{\ensuremath{\mathop{{}^{#1}\mathrm{log}(#2)}}}
% % \mathop{{}^{#1}\mathrm{log}}
% \newcommand{\logxfrac}[2]{\ensuremath{\frac{\log #2}{\log #1}}}
% 
\begin{document}
% \setcounter{chapter}{4}
% \chapter{Kwadratische verbanden}
\setcounter{chapter}{10}
\chapter{Populatie en steekproef}
\begin{antwoord}{1}
Uit een groep van 500 personen moeten 30 personen worden gekozen.
Het aantal personen in de gelaagde steekproef is dan als volgt:
\begin{itemize}
 \item minder dan \euro 5000: 1
 \\ 
 \[ \frac{30}{500} \cdot 24 = 1,44 \approx 1 \]
 \item tussen \euro 5000 en \euro 10000: 3 
 \\ 
 \[ \frac{30}{500} \cdot 48 = 2,88 \approx 3  \]
 \item tussen \euro 10000 en \euro 25000: 7 
 \\ 
 \[ \frac{30}{500} \cdot 110 = 6,6 \approx 7  \]
 \item tussen \euro 25000 en \euro 60000: 16
 \\ 
 \[ \frac{30}{500} \cdot 262 = 15,72  \approx 16 \]
 \item tussen \euro 60000 en \euro 100000: 2 
 \\ 
 \[ \frac{30}{500} \cdot 41 = 2,46  \approx 2  \]
 \item meer dan \euro 100000: 1
 \\ 
 \[ \frac{30}{500} \cdot 15 = 0,9  \approx 1  \]
 \end{itemize}
% \begin{align*}
% \text{Minder dan \euro 5000: } & \frac{30}{500} \cdot 24 = 1,44 \approx 1 \\
% \text{Tussen \euro 5000 en \euro 10000: } & \frac{30}{500} \cdot 48 = 2,88 \approx 3  \\
% \text{Tussen \euro 10000 en \euro 25000: } & \frac{30}{500} \cdot 110 = 6,6 \approx 7 \\
% \text{Tussen \euro 25000 en \euro 60000: } & \frac{30}{500} \cdot 262 = 15,72  \approx 16 \\
% \text{Tussen \euro 60000 en \euro 100000: } & \frac{30}{500} \cdot 41 = 2,46  \approx 2 \\
% \text{Meer dan \euro 100000: } & \frac{30}{500} \cdot 15 = 0,9  \approx 1  
% \end{align*}
\end{antwoord}

\begin{antwoord}{4}
\begin{enumerate}
 \item 
 \item 
    \emph{Bepaling november 2009}:

    November 2009 zit tussen de maanden oktober 2009 en januari 2010.
    \begin{align*}
    \text{Oktober 2009} &: 125 \\
    \text{November 2009} &: \text{ ?} \\
    \text{December 2009} &: \text{ ?} \\
    \text{Januari 2010} &: 175 \\
    \end{align*}

    Het verschil tussen de verkoop in januari 2010 en oktober 2009 is $175 - 125 = 50$ ton. Dit is een verschil van 50 ton in 3 maanden. Per maand, neemt de verkoop ongeveer $\frac{50}{3} = 16,6666\ldots$ ton toe.

    In november 2009 is de verkoop dus waarschijnlijk 
    \[
    125 + \frac{50}{3} \approx 142 \text{ tonnen}
    \]

    \bigskip
    \emph{Bepaling november 2010}:

    November 2010 zit tussen de maanden oktober 2010 en januari 2011.
    \begin{align*}
    \text{Oktober 2010} &: 290 \\
    \text{November 2010} &: \text{ ?} \\
    \text{December 2010} &: \text{ ?} \\
    \text{Januari 2011} &: 240 \\
    \end{align*}

    Het verschil tussen de verkoop in januari 2011 en oktober 2010 is $240 - 290 = - 50$ ton. Dit is een verschil van $-50$ ton in 3 maanden. Per maand, neemt de verkoop ongeveer $\frac{-50}{3} = - 16,6666\ldots$ ton toe.

    In november 2010 is de verkoop dus waarschijnlijk 
    \[
    290 + \frac{-50}{3} \approx 273 \text{ tonnen}
    \]
%  \item 
%  \item 
\end{enumerate}
\end{antwoord}


% 
% \setcounter{chapter}{7}
% \chapter{Logaritmen}
% % % \section{Vermenigvuldigen met machten}
\begin{antwoord}{2a}
Stap 0: $f(x) = x^2 + 5x - 14$, dus $a = 1$, $b = 5$ en $c = -14$.

\medskip
Stap 2:
\begin{align*}
\xtop &= \frac{-b}{2a} = \frac{-5}{2 \cdot 1} = -2,5 \\
\ytop &= \alert{(-2,5)}^2 + 5 \cdot \alert{(-2,5)} - 14\\ 
&= 6,25 + - 12,5- 14\\
&= -20,25\\
Top &: (-2,5; -20,25)
\end{align*}

\medskip
Stap 1: Snijpunten $x-as$:
\begin{align*}
x_1 &= \frac{-b + \sqrt{b^2 - 4ac}}{2a} \\
&= \frac{-5 + \sqrt{5^2 - 4\cdot 1 \cdot -14}}{2 \cdot 1} \\
&= \frac{-5 + \sqrt{25 - 4 \cdot -14}}{2} \\
&= \frac{-5 + \sqrt{25 + 56}}{2} \\
&= \frac{-5 + \sqrt{81}}{2} = 2\\
y_1 &= 0 \qquad \text{(snijpunt met $x$-as)} \\
S_1 &= (2,0)
\end{align*}
\begin{align*}
x_2 &= \frac{-b - \sqrt{b^2 - 4ac}}{2a} \\
&= \frac{-5 - \sqrt{5^2 - 4\cdot 1 \cdot -14}}{2 \cdot 1} \\
&= \frac{-5 - \sqrt{81}}{2} = - 7\\
y_2 &= 0 \qquad \text{(snijpunt met $x$-as)} \\
S_2 &= (-7,0)
\end{align*}

\medskip
Stap 3: Snijpunt $y$-as is op $(0, c)$, dus snijpunt $(0, -14)$.\\

\medskip
Stap 4: Spiegelen: top is in het rood (symmetrie-as)
\begin{wiskundetabel}{5}{x}{y}{-7 & \phantom{-33}  & \alert{-2,5} & 0 & 2}{ 0 & & \alert{-20,25} & -14 & 0}
\end{wiskundetabel}

We weten $(0, -14)$. Spiegelen langs de symmetrie-as geeft:
\begin{wiskundetabel}{5}{x}{y}{-7 & \alertb{-5}  & \alert{-2,5} & 0 & 2}{ 0 & \alertb{-14} & \alert{-20,25} & -14 & 0}
\end{wiskundetabel}

Stap 5: Random punten invullen als je nog geen 5 punten hebt. Hier is dit niet nodig. 

We krijgen:\\
\begin{tikzpicture}[scale=0.5]
\tkzInit[xmin=-10, xmax=5,ymax=5,ymin=-25]
\tkzGrid
    \tkzLabelX[orig=true,label options={font=\scriptsize}]
    \tkzLabelY[orig=false,label options={font=\scriptsize}]
    \tkzDrawX
    \tkzDrawY
    \tkzDefPoint(-7,0){A}
    \tkzDefPoint(-5,-14){B}
    \tkzDefPoint(-2.5,-20.25){C}
    \tkzDefPoint(0,-14){D}
    \tkzDefPoint(2,0){E}
    \tkzDrawPoints(A, B, C, D, E)
    \draw[scale=1, domain=-7.5:2.5,smooth,variable=\x,blue, line width = 1pt] plot ({\x},{\x*\x+5*\x-14});
\end{tikzpicture}
\end{antwoord}





\begin{antwoord}{3b}
Stap 0: $h(a) = a^2 + 7a + 12$, dus $a = 1$, $b = 7$ en $c = 12$.

\medskip
Stap 2:
\begin{align*}
\xtop &= \frac{- 7}{2 \cdot 1} = \frac{-7}{2} = -3,5 \\
\ytop &= \alert{(-3,5)}^2 + 7 \cdot \alert{(-3,5)} + 12\\ 
&= 12,25 + - 24,5 + 12\\
&= -0,25\\
Top &: (-3,5; -0,25)
\end{align*}

\medskip
Stap 1: Snijpunten $x-as$:
\begin{align*}
x_1 &= \frac{-b + \sqrt{b^2 - 4ac}}{2a} \\
&= \frac{-7 + \sqrt{7^2 - 4\cdot 1 \cdot 12}}{2 \cdot 1} \\
&= -3\\
y_1 &= 0 \qquad \text{(snijpunt met $x$-as)} \\
S_1 &= (-3,0)
\end{align*}
\begin{align*}
x_2 &= \frac{-b - \sqrt{b^2 - 4ac}}{2a} \\
&= \frac{-7 - \sqrt{7^2 - 4\cdot 1 \cdot 12}}{2 \cdot 1} \\
&= -4\\
y_2 &= 0 \qquad \text{(snijpunt met $x$-as)} \\
S_2 &= (-4,0)
\end{align*}

\medskip
Stap 3: Snijpunt $y$-as is op $(0, c)$, dus snijpunt $(0, 12)$.\\

\medskip
Stap 4: Spiegelen: top is in het rood (symmetrie-as)
\begin{wiskundetabel}{5}{x}{y}{\phantom{eee} & -4 & \alert{-3,5}  &  -3 & 0}{ & 0 & \alert{-0,25} & 0 & 12}
\end{wiskundetabel}

We weten $(0, 12)$. Spiegelen langs de symmetrie-as geeft:
\begin{wiskundetabel}{5}{x}{y}{\phantom{eee} & -4 & \alert{-3,5}  &  -3 & 0}{ & 0 & \alert{-0,25} & 0 & 12}
\end{wiskundetabel}


Stap 5: Random punten invullen als je nog geen 5 punten hebt. Hier is dit niet nodig. 

We krijgen:\\
\begin{tikzpicture}[scale=0.5]
\tkzInit[xmin=-10, xmax=5,ymax=5,ymin=-25]
\tkzGrid
    \tkzLabelX[orig=true,label options={font=\scriptsize}]
    \tkzLabelY[orig=false,label options={font=\scriptsize}]
    \tkzDrawX
    \tkzDrawY
    \tkzDefPoint(-7,0){A}
    \tkzDefPoint(-5,-14){B}
    \tkzDefPoint(-2.5,-20.25){C}
    \tkzDefPoint(0,-14){D}
    \tkzDefPoint(2,0){E}
    \tkzDrawPoints(A, B, C, D, E)
    \draw[scale=1, domain=-7.5:2.5,smooth,variable=\x,blue, line width = 1pt] plot ({\x},{\x*\x+7*\x+12});
\end{tikzpicture}
\end{antwoord}



























\newpage
\begin{antwoord}{3}
\begin{enumerate}
\item $g(x) = x^2 - 2x +3$, dus $a = 1$, $b = -2$ en $c = 3$.

\medskip
$\xtop = \frac{-b}{2a} = \frac{2}{2 * 1} = 1 $
\begin{align*}
\ytop &= \alert{1}^2 - 2 \cdot \alert{1} + 3\\ 
&= 1 - 2 +3\\
&= 2\\
T &= (1, 2)
\end{align*}
% 
% \medskip
% Snijpunten $x-as$:
% \begin{align*}
% x_1 &= \frac{-b + \sqrt{b^2 - 4ac}}{2a} \\
% &= \frac{-5 + \sqrt{5^2 - 4\cdot 1 \cdot -14}}{2 \cdot 1} \\
% &= \frac{-5 + \sqrt{25 - 4 \cdot -14}}{2} \\
% &= \frac{-5 + \sqrt{25 + 56}}{2} \\
% &= \frac{-5 + \sqrt{81}}{2} = 2\\
% y_1 &= 0 \qquad \text{(snijpunt met $x$-as)} \\
% S_1 &= (2,0)
% \end{align*}
% \begin{align*}
% x_2 &= \frac{-b - \sqrt{b^2 - 4ac}}{2a} \\
% &= \frac{-5 - \sqrt{5^2 - 4\cdot 1 \cdot -14}}{2 \cdot 1} \\
% &= \frac{-5 - \sqrt{81}}{2} = - 7\\
% y_2 &= 0 \qquad \text{(snijpunt met $x$-as)} \\
% S_2 &= (-7,0)
% \end{align*}
% 
% \item $5^5$
% \item $27^{15}$
% \item $4^{12}$
\end{enumerate}
\end{antwoord}

% \begin{antwoord}{15}
\begin{enumerate}
 \item[a)]
\begin{align*}
  3^x &= 200 \\
  x &= \logx{3}{200} \\
  &= \logxfrac{3}{200} = 4,822736302
\end{align*}
 \item[b)]
\begin{align*}
  4^{2x} &= 345 \\
  2x &= \logx{4}{345} \\
  2x &= \logxfrac{4}{345} = 4,215226276\\
  x &= 2,10761318
  \end{align*}
\end{enumerate}
\end{antwoord}

\begin{antwoord}{16}
\begin{enumerate}
 \item[a)]
\begin{align*}
  x &= \logx{4}{6} \\
  &= \logxfrac{4}{6} = 1,29248125
\end{align*}
 \item[b)]
\begin{align*}
  1,87^x &= 48,9 \\
  x &= \logx{1,87}{48,9} \\
  &= \logxfrac{1,87}{48,9} \\
  &= 6,214313109
\end{align*}

 \item[c)]
 \begin{align*}
 \logx{8}{56} &= x \\
 x &= \logxfrac{8}{56} = 1,935784974  \\
 \end{align*}
 \item[d)]
 \begin{align*}
 2,8^{3x} &= 68 \\
 3x &= \logx{2,8}{68} = \logxfrac{2,8}{68} = 4,098123671\\
 x &= \frac{4,098123671}{3} = 1,366041224
 \end{align*}
 \item[e)]
 \begin{align*}
\logx{5}{98^2} &= x = \logx{5}{9604}  \\
 x &= \logxfrac{5}{9604} = 5,697600937
 \end{align*}
 \item[f)]
 \begin{align*}
x &= \logx{0,3}{0,1}  \\
&= \logxfrac{0,3}{0,1}  \\
&= 1,912489289
\end{align*}
 \item[g)]
 \begin{align*}
 3^x &= 50 \\
x &= \logx{3}{50}  \\
&= \logxfrac{3}{50}  \\
&= 3,560876795
\end{align*}
 \item[h)]
 \begin{align*}
x &= \logx{0,9}{0,1}  \\
&= \logxfrac{0,9}{0,1}  \\
&= 21,85434533
\end{align*}
 \item[i)]
 \begin{align*}
x &= \logx{-4}{56}
\end{align*}
Kan niet
 \item[j)]
 \begin{align*}
x &= \logx{333}{0,1}\\
 &= \logxfrac{333}{0,1}\\
 &= -0,396440875
\end{align*}
 \item[k)]
 \begin{align*}
45 \cdot 3,85^x &= 567,0 \cdot 10^3 \\
3,85^x &= \frac{567,0 \cdot 10^3}{45} = 12600 \\
x &= \logx{3,85}{12600} = \logxfrac{3,85}{12600} = 7,00366453\\
\end{align*}
 \item[l)]
 \begin{align*}
x &= \logx{4}{-8}
\end{align*}
Kan niet
 \end{enumerate}
\end{antwoord}

\begin{antwoord}{17}
\begin{enumerate}
  \item[b)] $N(t) = N(0) \cdot g^t$ met
  \begin{align*}
    N(0) &= 250 \\
    g &= 2 \\
    t &= \text{Aantal perioden van 10 uur sinds 08:00}
  \end{align*}
  Dit geeft \[N(t) = 250 \cdot 2^t\]
  \item[a)] Om 18:00 zijn er 10 uren voorbij sinds 08:00. Dit geeft $t = 1$ (want er is 1 periode van 10 uur voorbij).
  Om 18:00 zijn er \[N(1) = 250 \cdot 2^1 = 500\] bacteri\"en per mg.

  \bigskip
  Om 22:00 zijn er 14 uren voorbij sinds 08:00. Dit geeft $t = 1,4$ (want $\frac{14}{10} = 1,4$).
  Om 22:00 zijn er \[N(1) = 250 \cdot 2^{1,4} = 659,7539554 \approx 660 \] bacteri\"en per mg.
  \item[c)]
  \begin{align*}
    N(t) &= 250 \cdot 2^t \\
    20000 &= 250 \cdot 2^t \\
    80 &= 2^t \\
    t &= \logx{2}{80}\\
    &= \logxfrac{2}{80} \approx 6,321928095
    \end{align*}
    Dit betekent dat de salade na $6,321928095 \cdot 10 = 63,21928095$ uren moet worden weggegooid.

    \medskip
    Omrekenen naar minuten: Een uur heeft 60 minuten, dus $0,21928095 $ uren zijn $0,21928095 \cdot 60 = 13,156857$ minuten.

    \bigskip
    Na 63 uren en \textbf{14} minuten kun je de salade weggooien (\'e\'en minuut verder en dan zit je over de grens).
\end{enumerate}
\end{antwoord}

\begin{antwoord}{18}
\begin{align*}
N(t) &= N(0) \cdot g^t \\ 
N(0) &= \SI{120}{\litre}\\
g &= 3\\
t &= \text{aantal perioden van 30 minuten vanaf 09:30}\\
\end{align*}
\[
 N(t) = 120 \cdot 3^t 
\]
\begin{enumerate}
\item Om 12:00 zijn er 5 perioden van 30 minuten voorbij sinds 09:30. Dus $t=5$.
Dit geeft
\begin{align*}
  N(t) &= 120 \cdot 3^5\\ 
  &= 29160 
\end{align*}
\item 
\[
 N(t) = 120 \cdot 3^t 
\]
\item 
\begin{align*}
120 \cdot 3^t &= 1 \\
 3^t &= 0,0083333333\ldots \\
 t &= \logx{3}{0,0083333333\ldots}\\
 &= -4,357762781
\end{align*}
$- 4$ perioden betekent dat je $2$~uur terug in de tijd gaat (van 09:30 terug naar 07:30). $-0,357762781$ perioden van 30 minuten zijn $- 0,357762781 \cdot 30 = -10,73288344$ minuten. 

We gaan dus 2 uur en 11 minuten terug in de tijd vanaf 09:30.
Dit betekent dat de wolk is ontstaan om 07:19.
\item
De wolk is \SI{5}{\litre} om 09:00. Om 11:00, dus 4 perioden erna, is de wolk \SI{60}{\litre}. Na 4 perioden is een gaswolk $g^4$ keer zo groot. Dit betekent: 
\begin{align*}
 5 \cdot g^4 &= 60 \\
 g^4 &= \frac{60}{5} = 12 \\
 g &= \sqrt[4]{12}\\
 &= 1,861209718
\end{align*}
De groeifactor is dus $1,861209718$.
\end{enumerate}
\end{antwoord}


\begin{antwoord}{19}
\begin{align*}
N(t) &= N(0) \cdot g^t \\ 
N(0) &= \SI{0,05}{\mm}\\
g &= 2\\
t &= \text{aantal keren knippen en stapelen}\\
\end{align*}
Dit geeft 
\[N(t) = 0,05 \cdot 2^t
\]
\begin{enumerate}
  \item 
  \begin{align*}
   N(3) &= 0,05 \cdot 2^3 = 0,40 \\
      N(5) &= 0,05 \cdot 2^5 = 1,6 \\
         N(10) &= 0,05 \cdot 2^{10} = 51,2 \\
  \end{align*}
  \item 
  \begin{align*}
N(t) &= 0,05 \cdot 2^t
  \end{align*}
\item \SI{3}{\m} hoog: \SI{3000}{\mm}.
\begin{align*}
0,05 \cdot 2^t  &= 3000\\
2^t  &= 60000\\
t  &= \logx{2}{60000}\\
&= \logxfrac{2}{60000}\\
t&= 15,87267488
\end{align*}
Na 16 keren stapelen past de stapel niet meer op de kamer.
\item \SI{3,844e8}{\m} hoog: \SI{3,488e11}{\mm}.
\begin{align*}
0,05 \cdot 2^t  &= \num{3,488e11}\\
2^t  &= \num{6,9622e12}\\
t  &= \logx{2}{\num{6,9622e12}}\\
&= \logxfrac{2}{\num{6,9622e12}}\\
t&= 42,6626804
\end{align*}
Na 43 keren stapelen bereik je de maan.


 \end{enumerate}
\end{antwoord}


% 
% % \setcounter{chapter}{12}
% % \chapter{De Gausskromme}
% % \section{Vermenigvuldigen met machten}
\begin{antwoord}{2a}
Stap 0: $f(x) = x^2 + 5x - 14$, dus $a = 1$, $b = 5$ en $c = -14$.

\medskip
Stap 2:
\begin{align*}
\xtop &= \frac{-b}{2a} = \frac{-5}{2 \cdot 1} = -2,5 \\
\ytop &= \alert{(-2,5)}^2 + 5 \cdot \alert{(-2,5)} - 14\\ 
&= 6,25 + - 12,5- 14\\
&= -20,25\\
Top &: (-2,5; -20,25)
\end{align*}

\medskip
Stap 1: Snijpunten $x-as$:
\begin{align*}
x_1 &= \frac{-b + \sqrt{b^2 - 4ac}}{2a} \\
&= \frac{-5 + \sqrt{5^2 - 4\cdot 1 \cdot -14}}{2 \cdot 1} \\
&= \frac{-5 + \sqrt{25 - 4 \cdot -14}}{2} \\
&= \frac{-5 + \sqrt{25 + 56}}{2} \\
&= \frac{-5 + \sqrt{81}}{2} = 2\\
y_1 &= 0 \qquad \text{(snijpunt met $x$-as)} \\
S_1 &= (2,0)
\end{align*}
\begin{align*}
x_2 &= \frac{-b - \sqrt{b^2 - 4ac}}{2a} \\
&= \frac{-5 - \sqrt{5^2 - 4\cdot 1 \cdot -14}}{2 \cdot 1} \\
&= \frac{-5 - \sqrt{81}}{2} = - 7\\
y_2 &= 0 \qquad \text{(snijpunt met $x$-as)} \\
S_2 &= (-7,0)
\end{align*}

\medskip
Stap 3: Snijpunt $y$-as is op $(0, c)$, dus snijpunt $(0, -14)$.\\

\medskip
Stap 4: Spiegelen: top is in het rood (symmetrie-as)
\begin{wiskundetabel}{5}{x}{y}{-7 & \phantom{-33}  & \alert{-2,5} & 0 & 2}{ 0 & & \alert{-20,25} & -14 & 0}
\end{wiskundetabel}

We weten $(0, -14)$. Spiegelen langs de symmetrie-as geeft:
\begin{wiskundetabel}{5}{x}{y}{-7 & \alertb{-5}  & \alert{-2,5} & 0 & 2}{ 0 & \alertb{-14} & \alert{-20,25} & -14 & 0}
\end{wiskundetabel}

Stap 5: Random punten invullen als je nog geen 5 punten hebt. Hier is dit niet nodig. 

We krijgen:\\
\begin{tikzpicture}[scale=0.5]
\tkzInit[xmin=-10, xmax=5,ymax=5,ymin=-25]
\tkzGrid
    \tkzLabelX[orig=true,label options={font=\scriptsize}]
    \tkzLabelY[orig=false,label options={font=\scriptsize}]
    \tkzDrawX
    \tkzDrawY
    \tkzDefPoint(-7,0){A}
    \tkzDefPoint(-5,-14){B}
    \tkzDefPoint(-2.5,-20.25){C}
    \tkzDefPoint(0,-14){D}
    \tkzDefPoint(2,0){E}
    \tkzDrawPoints(A, B, C, D, E)
    \draw[scale=1, domain=-7.5:2.5,smooth,variable=\x,blue, line width = 1pt] plot ({\x},{\x*\x+5*\x-14});
\end{tikzpicture}
\end{antwoord}





\begin{antwoord}{3b}
Stap 0: $h(a) = a^2 + 7a + 12$, dus $a = 1$, $b = 7$ en $c = 12$.

\medskip
Stap 2:
\begin{align*}
\xtop &= \frac{- 7}{2 \cdot 1} = \frac{-7}{2} = -3,5 \\
\ytop &= \alert{(-3,5)}^2 + 7 \cdot \alert{(-3,5)} + 12\\ 
&= 12,25 + - 24,5 + 12\\
&= -0,25\\
Top &: (-3,5; -0,25)
\end{align*}

\medskip
Stap 1: Snijpunten $x-as$:
\begin{align*}
x_1 &= \frac{-b + \sqrt{b^2 - 4ac}}{2a} \\
&= \frac{-7 + \sqrt{7^2 - 4\cdot 1 \cdot 12}}{2 \cdot 1} \\
&= -3\\
y_1 &= 0 \qquad \text{(snijpunt met $x$-as)} \\
S_1 &= (-3,0)
\end{align*}
\begin{align*}
x_2 &= \frac{-b - \sqrt{b^2 - 4ac}}{2a} \\
&= \frac{-7 - \sqrt{7^2 - 4\cdot 1 \cdot 12}}{2 \cdot 1} \\
&= -4\\
y_2 &= 0 \qquad \text{(snijpunt met $x$-as)} \\
S_2 &= (-4,0)
\end{align*}

\medskip
Stap 3: Snijpunt $y$-as is op $(0, c)$, dus snijpunt $(0, 12)$.\\

\medskip
Stap 4: Spiegelen: top is in het rood (symmetrie-as)
\begin{wiskundetabel}{5}{x}{y}{\phantom{eee} & -4 & \alert{-3,5}  &  -3 & 0}{ & 0 & \alert{-0,25} & 0 & 12}
\end{wiskundetabel}

We weten $(0, 12)$. Spiegelen langs de symmetrie-as geeft:
\begin{wiskundetabel}{5}{x}{y}{\phantom{eee} & -4 & \alert{-3,5}  &  -3 & 0}{ & 0 & \alert{-0,25} & 0 & 12}
\end{wiskundetabel}


Stap 5: Random punten invullen als je nog geen 5 punten hebt. Hier is dit niet nodig. 

We krijgen:\\
\begin{tikzpicture}[scale=0.5]
\tkzInit[xmin=-10, xmax=5,ymax=5,ymin=-25]
\tkzGrid
    \tkzLabelX[orig=true,label options={font=\scriptsize}]
    \tkzLabelY[orig=false,label options={font=\scriptsize}]
    \tkzDrawX
    \tkzDrawY
    \tkzDefPoint(-7,0){A}
    \tkzDefPoint(-5,-14){B}
    \tkzDefPoint(-2.5,-20.25){C}
    \tkzDefPoint(0,-14){D}
    \tkzDefPoint(2,0){E}
    \tkzDrawPoints(A, B, C, D, E)
    \draw[scale=1, domain=-7.5:2.5,smooth,variable=\x,blue, line width = 1pt] plot ({\x},{\x*\x+7*\x+12});
\end{tikzpicture}
\end{antwoord}



























\newpage
\begin{antwoord}{3}
\begin{enumerate}
\item $g(x) = x^2 - 2x +3$, dus $a = 1$, $b = -2$ en $c = 3$.

\medskip
$\xtop = \frac{-b}{2a} = \frac{2}{2 * 1} = 1 $
\begin{align*}
\ytop &= \alert{1}^2 - 2 \cdot \alert{1} + 3\\ 
&= 1 - 2 +3\\
&= 2\\
T &= (1, 2)
\end{align*}
% 
% \medskip
% Snijpunten $x-as$:
% \begin{align*}
% x_1 &= \frac{-b + \sqrt{b^2 - 4ac}}{2a} \\
% &= \frac{-5 + \sqrt{5^2 - 4\cdot 1 \cdot -14}}{2 \cdot 1} \\
% &= \frac{-5 + \sqrt{25 - 4 \cdot -14}}{2} \\
% &= \frac{-5 + \sqrt{25 + 56}}{2} \\
% &= \frac{-5 + \sqrt{81}}{2} = 2\\
% y_1 &= 0 \qquad \text{(snijpunt met $x$-as)} \\
% S_1 &= (2,0)
% \end{align*}
% \begin{align*}
% x_2 &= \frac{-b - \sqrt{b^2 - 4ac}}{2a} \\
% &= \frac{-5 - \sqrt{5^2 - 4\cdot 1 \cdot -14}}{2 \cdot 1} \\
% &= \frac{-5 - \sqrt{81}}{2} = - 7\\
% y_2 &= 0 \qquad \text{(snijpunt met $x$-as)} \\
% S_2 &= (-7,0)
% \end{align*}
% 
% \item $5^5$
% \item $27^{15}$
% \item $4^{12}$
\end{enumerate}
\end{antwoord}

% % \begin{antwoord}{1}
\begin{enumerate}
 \item 68\%, dus we hebben een 1-sigma gebied nodig:
 \begin{align*}
  \mu - 1\sigma &= 37 - 1 \cdot 3 = 34 \\
  \mu + 1\sigma &= 37 + 1 \cdot 3 = 40 
 \end{align*}
 \item 95\%, dus we hebben een 2-sigma gebied nodig:
 \begin{align*}
  \mu - 2\sigma &= 37 - 2 \cdot 3 = 31 \\
  \mu + 2\sigma &= 37 + 2 \cdot 3 = 43 
 \end{align*}
 \item bijna alle, dus 99,7\%, dus we hebben een 3-sigma gebied nodig:
 \begin{align*}
  \mu - 3\sigma &= 37 - 3 \cdot 3 = 28 \\
  \mu + 3\sigma &= 37 + 3 \cdot 3 = 46 
 \end{align*}
 \end{enumerate} 
\end{antwoord}

% % \chapter{Steekproeven}
% % \begin{antwoord}{1}
 Meetwaarden: $0,788$ - $0,824$ - $0,612$ - $0,812$ - $0,780$
 
 Stap 1, volgorde: $0,612$ - $0,780$ - $0,788$ - $0,812$ - $0,824$  \\
 Stap 2: $0,780 - 0,612 = 0,168$\\
 Stap 3: $0,824 - 0,812 = 0,012$  
 
 Stap 4:
 Verdachte waarde: $0,612$.\\
 $\text{Spreidingsbreedte} = 0,824 - 0,612 = 0,212$\\
 \[
 P = \frac{0,168}{0,212} = 0,79
 \]
 
 Tabel ($n = 5$): $u.c. = 0,78$. We zien dat $0,79$ groter is dan $0,78$.
 Dus $0,612$ is een uitschieter.

 \bigskip
 Stap 5: Meetwaarden: 
 $0,780$ - $0,788$ - $0,812$ - $0,824$\\
 
 Stap 2: $0,788 - 0,780 = 0,008$\\
 Stap 3: $0,824 - 0,812 = 0,012$  
 
 Stap 4:
 Verdachte waarde: $0,824$.\\
 $\text{Spreidingsbreedte} = 0,824 - 0,780 = 0,044$\\
 \[
 P = \frac{0,012}{0,044} = 0,27
 \]
 
 Tabel ($n = 5$): $u.c. = 0,88$. We zien dat $0,27 < 0,88$.\\
 Dus geen uitschieters!
 
 
 \end{antwoord} 

% s
\end{document}
