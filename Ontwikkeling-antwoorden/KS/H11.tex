\begin{antwoord}{1}
Uit een groep van 500 personen moeten 30 personen worden gekozen.
Het aantal personen in de gelaagde steekproef is dan als volgt:
\begin{itemize}
 \item minder dan \euro 5000: 1
 \\ 
 \[ \frac{30}{500} \cdot 24 = 1,44 \approx 1 \]
 \item tussen \euro 5000 en \euro 10000: 3 
 \\ 
 \[ \frac{30}{500} \cdot 48 = 2,88 \approx 3  \]
 \item tussen \euro 10000 en \euro 25000: 7 
 \\ 
 \[ \frac{30}{500} \cdot 110 = 6,6 \approx 7  \]
 \item tussen \euro 25000 en \euro 60000: 16
 \\ 
 \[ \frac{30}{500} \cdot 262 = 15,72  \approx 16 \]
 \item tussen \euro 60000 en \euro 100000: 2 
 \\ 
 \[ \frac{30}{500} \cdot 41 = 2,46  \approx 2  \]
 \item meer dan \euro 100000: 1
 \\ 
 \[ \frac{30}{500} \cdot 15 = 0,9  \approx 1  \]
 \end{itemize}
% \begin{align*}
% \text{Minder dan \euro 5000: } & \frac{30}{500} \cdot 24 = 1,44 \approx 1 \\
% \text{Tussen \euro 5000 en \euro 10000: } & \frac{30}{500} \cdot 48 = 2,88 \approx 3  \\
% \text{Tussen \euro 10000 en \euro 25000: } & \frac{30}{500} \cdot 110 = 6,6 \approx 7 \\
% \text{Tussen \euro 25000 en \euro 60000: } & \frac{30}{500} \cdot 262 = 15,72  \approx 16 \\
% \text{Tussen \euro 60000 en \euro 100000: } & \frac{30}{500} \cdot 41 = 2,46  \approx 2 \\
% \text{Meer dan \euro 100000: } & \frac{30}{500} \cdot 15 = 0,9  \approx 1  
% \end{align*}
\end{antwoord}

\begin{antwoord}{4}
\begin{enumerate}
 \item 
 \item 
    \emph{Bepaling november 2009}:

    November 2009 zit tussen de maanden oktober 2009 en januari 2010.
    \begin{align*}
    \text{Oktober 2009} &: 125 \\
    \text{November 2009} &: \text{ ?} \\
    \text{December 2009} &: \text{ ?} \\
    \text{Januari 2010} &: 175 \\
    \end{align*}

    Het verschil tussen de verkoop in januari 2010 en oktober 2009 is $175 - 125 = 50$ ton. Dit is een verschil van 50 ton in 3 maanden. Per maand, neemt de verkoop ongeveer $\frac{50}{3} = 16,6666\ldots$ ton toe.

    In november 2009 is de verkoop dus waarschijnlijk 
    \[
    125 + \frac{50}{3} \approx 142 \text{ tonnen}
    \]

    \bigskip
    \emph{Bepaling november 2010}:

    November 2010 zit tussen de maanden oktober 2010 en januari 2011.
    \begin{align*}
    \text{Oktober 2010} &: 290 \\
    \text{November 2010} &: \text{ ?} \\
    \text{December 2010} &: \text{ ?} \\
    \text{Januari 2011} &: 240 \\
    \end{align*}

    Het verschil tussen de verkoop in januari 2011 en oktober 2010 is $240 - 290 = - 50$ ton. Dit is een verschil van $-50$ ton in 3 maanden. Per maand, neemt de verkoop ongeveer $\frac{-50}{3} = - 16,6666\ldots$ ton toe.

    In november 2010 is de verkoop dus waarschijnlijk 
    \[
    290 + \frac{-50}{3} \approx 273 \text{ tonnen}
    \]
%  \item 
%  \item 
\end{enumerate}
\end{antwoord}

