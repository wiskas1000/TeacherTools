\documentclass{beamer}

\usepackage[dutch]{babel}
\usepackage{xcolor}
\usepackage{booktabs}
\usetheme{Singapore}
\useinnertheme[shadow]{rounded}
\usecolortheme{rose}
\definecolor{TUD}{cmyk}{1,0,0,0}


\providecommand{\lh}{\ensuremath{\left(}}
\providecommand{\rh}{\ensuremath{\right)}}
\providecommand{\lbr}{\ensuremath{\left\lbrace}}
\providecommand{\rbr}{\ensuremath{\right\rbrace}}
\providecommand{\matlab}{MATLAB\textregistered}

% \providecommand{\pr}[1]{\ensuremath{\mathbb{P}\left(#1\right)}}
% \providecommand{\E}[2][]{\ensuremath{\mathbb{E}_{#1}\left[#2\right]}}
% \providecommand{\isdef}{\ensuremath{\stackrel{\mathrm{def}}{=} }}
\providecommand{\abs}[1]{\left| #1 \right|}


% % \theoremstyle{theorem}
% \newtheorem{stelling}{Stelling}
% \newtheorem*{gevolg}{Gevolg}
% \theoremstyle{definition}
% \newtheorem{definitie}{Definitie}


\title{Wiskunde BMA 2A/2B/2C}
% \subtitle{Sewi}
\author{Wikash Sewlal}
\institute{Techniek College Rotterdam}
\date{2019 --- 2020}
% \date{11 augustus 2010}

% \AtBeginSection[]
% {
%   \begin{frame}<beamer>
%     \tableofcontents[currentsection]
%   \end{frame}
% }

\begin{document}
\frame{\titlepage}
% \input{./kennismaking.tex}
% \input{./wis-bma-2bc-intro.tex}
% \input{./wis-bma-2bc-benodigdheden.tex}
% % \input{./fipronil.tex}
% % \frame{\tableofcontents}
% % \input{./4h_rekenen.tex}
% \input{./werkwijze.tex}
% \input{./regels.tex}

\begin{frame}
\frametitle{Antwoorden}
\begin{block}{Antwoord 2b}
\begin{description}
 \item[Stap 0:] $f(x) = - 2 x^{2} + 3 x + 2 $ met $a =  -2 $, $b =  3 $ en $c =  2 $.
\item[Stap 2:] $\frac{-b}{2a} = \frac{- 3 }{2 \cdot  -2 } =  0,75$\\
Top = $( 0,75 ; 3,125 )$
\item[Stap 1:]
Oplossingen: $x_1 =  - 0,5$. Dus $(-0,5;0)$.\\
Oplossingen: $x_2 =  2$. Dus $(2,0)$.\\
\item[Stap 3:] Snijpunt $y$-as is op $(0,c)$, dus snijpunt $(0, 2 )$\\
\item[Stap 4:] Spiegelen met de $y$-as: punt $(2 \cdot x_{\text{top}},c)$, dus spiegelpunt $( 1,5 ; 2 )$
\end{description}
\end{block}
\end{frame}

\begin{frame}
\frametitle{Antwoorden}
\begin{block}{Antwoord 2c}
\begin{description}
 \item[Stap 0:] $f(x) = 2 x^{2} - 4 x + 5 $ met $a =  2 $, $b =  -4 $ en $c =  5 $.
% Stap 2: $\frac{-b}{2a} = \frac{- 3 }{2 \cdot  -2 } =  0,75$\\
% Top = $( 0,75 ; 3,125 )$
\item[Stap 2:] $\frac{-b}{2a} = \frac{- -4 }{2 \cdot  2 } =  1$\\
Top = $( 1, 3 )$
\item[Stap 1:] $D = b^2 - 4 \cdot a \cdot c = (-4)^2 - 4 \cdot 2 \cdot 5 = 16 - 40 = -24$. Dus er zijn geen oplossingen met de $x$-as.
\item[Stap 3:] Snijpunt $y$-as is op $(0,c)$, dus snijpunt $(0, 5 )$\\
\item[Stap 4:] Spiegelen met de $y$-as: punt $(2 \cdot x_{\text{top}},c)$, dus spiegelpunt 
$( 2, 5 )$
\item[Stap 5:] Nodig! We moeten 2 random punten invullen.
\end{description}
\end{block}
\end{frame}


\begin{frame}
\frametitle{Antwoorden}
\begin{block}{Antwoord 2d}
\begin{description}
 \item[Stap 0:] $f(x) = - 3,43 x^{2} + 1,76 x - 7,574 $ met $a =  -3,43 $, $b =  1,76 $ en $c =  -7,574 $.
\item[Stap 2:] $\frac{-b}{2a} = \frac{- 1,76 }{2 \cdot  -3,43 } =  0,25656$\\
Top = $( 0,25656 ; -7,3482 )$
\item[Stap 1:] $D = b^2 - 4 \cdot a \cdot c = (-4)^2 - 4 \cdot 2 \cdot 5 = 16 - 40 = -24$. Dus er zijn geen oplossingen met de $x$-as.

\item[Stap 3:] Stap 3: Snijpunt $y$-as is op $(0,c)$, dus snijpunt $(0; -7,574 )$
\item[Stap 4:] Stap 4: Spiegelen met de $y$-as: punt $(2 \cdot x_{\text{top}},c)$, dus spiegelpunt $( 0,51312 ; -7,574 )$
\item[Stap 5:] Nodig! We moeten 2 random punten invullen.
\end{description}
\end{block}
\end{frame}


\end{document}
