\section{Hoofdstuk 11: Populatie en steekproef}
\begin{frame}{H11: Populatie en steekproef}
 \begin{block}{Lesdoel}
  \begin{itemize}
   \item Lezen: blz 59 t/m 62
   \item Lezen: blz 63 t/m 65
   \item Maken: H11 opgaven 1, 2, 4 (, 3)
  \end{itemize}
 \end{block}
\end{frame}

\begin{frame}{H11: Populatie en steekproef}

Filmpje~1
\url{https://www.youtube.com/watch?v=YJvWxZJKB4s}

Filmpje~2
\url{https://www.youtube.com/watch?v=oFX4g24taIw}

\bigskip
Vraag: Kun je elk ei testen op fipronil?
\end{frame}

\begin{frame}{H11: Populatie en steekproef}

Albert Heijn roept alle boerenkaas $30+$ terug
\url{https://www.rtlnieuws.nl/nieuws/nederland/artikel/4989951/albert-heijn-terugroepactie}

\bigskip
Vraag: Wordt elke verpakking getest?
\end{frame}

\begin{frame}{H11: Populatie en steekproef}
 \begin{block}{Steekproef en populatie}
  \begin{description}
   \item[Populatie] De volledige groep die je wilt onderzoeken.
   \\ Bijv. Het rivierwater van de Maas in Rotterdam
   \\ Bijv. Alle eieren van een boerderij
   \item[Steekproef] Deel van de populatie.
   \\ Bijv. Watermonster uit de rivier
   \\ Bijv. Een aantal dozen eieren van deze boerderij
  \end{description}
 \end{block}
\end{frame}


\begin{frame}{H11: Populatie en steekproef}
 \begin{block}{Steekproeven (11.1 t/m 11.4)}
  \begin{itemize}
   \item Lezen: blz 59 t/m 62
        \\ Tot en met paragraaf 11.4 lezen.
  \end{itemize}
 \end{block}
\end{frame}

\begin{frame}{H11: Populatie en steekproef}
 \begin{block}{5 belangrijke begrippen bij het onderwerp steekproeven}
  \begin{itemize}
   \item representatief
%    \item 4 typen steekproeven
%     \begin{itemize}   
    \item aselect
    \item loting
    \item gelaagde steekproef
    \item systematisch
%     \end{itemize}
  \end{itemize}
 \end{block}
 
 Let op! Deze begrippen sluiten elkaar niet uit. 
%  Ze hoeven niet exclusief van elkaar gebruikt te worden
 Je kunt ze combineren (opgave 5).
\end{frame}

\begin{frame}{H11: Populatie en steekproef}
 \begin{block}{Tabellen en grafieken (11.5)}
  \begin{itemize}
   \item Lezen: blz 63 t/m 65
   \item Maken: H11 opgaven 1, 2, 4 (, 3)
  \end{itemize}
 \end{block}
\end{frame}

\begin{frame}{H11: Populatie en steekproef}
 \begin{block}{Huiswerk}
  \begin{itemize}
   \item Lezen: blz 59 t/m 62
   \item Lezen: blz 63 t/m 65
   \item Maken: H11 opgaven 1, 2, 4 (, 3)
   \item Verschil kennen tussen een populatie en een steekproef.
   \item De 5 belangrijke begrippen bij het onderwerp steekproeven kennen en begrijpen.
  \end{itemize}
 \end{block}
\end{frame}
